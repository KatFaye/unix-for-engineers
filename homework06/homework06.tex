\documentclass[letterpaper]{article}

\title{Homework 06: Diversity in Computer Science}
\date{3/21/2016}
\author{Kat Herring \\ kherring@nd.edu}

\usepackage{graphicx}
\usepackage{hyperref}
\usepackage[margin=1in]{geometry}

\begin{document}

\maketitle

This document provides a summary of gender and racial diversity in Computer Science.

\section{Overview}

Using shell scripts and gnuplot, I found that Notre Dame's Computer Science department is more demographically diverse than the tech industry as a whole. However, Caucasians and males are still disproportionately over-represented in the deparment.

\section{Methodology}

In order to process the provided data, I wrote a shell script that used awk in order to create a series of associative arrays of the racial and gender demographics at Notre Dame. The totals are then printed and saved into the file demographics.dat.

Demographics.dat is then split into two files, gender.dat and race.dat, which contain the corresponding statistical information.

One of the biggest challenges was in using awk to parse the data in the format that I wanted, as associative arrays are still fairly new to me.

\section{Analysis}

The demographic statistics for gender in the Notre Dame Computer Science department are displayed in Figure \ref{fig:gender}.

Overall, the data depicts a dramatic increase in the size of the Computer Science department as a whole, with the gender balance increasing slightly during the past five years. \ref{tbl:results}

\begin{figure}[h]
\centering
\includegraphics[width=5in]{gender.png}
\caption{Histogram of Gender Diversity}
\label{fig:gender}
\end{figure}


\begin{table}[h!]
    \centering
    \begin{tabular}{r||c}
    Year	& Percentage of Female Students\\
    \hline
    2013	& 22\% \\
    2014	& 21\% \\
    2015	& 21\% \\
    2016	& 24\% \\
    2017 & 28\% \\
    2018 & 28\% \\
    \end{tabular}
    \caption{Gender balance in CSE over past five years}
    \label{tbl:results}
\end{table}

The demographic statistics for race in the Notre Dame Computer Science department are displayed in Figure \ref{fig:race}. While the specific breakdown of minorities varies slightly from year to year, the department as a whole as in face gotten less racially diverse and the percentage of students that are Caucasian.

\begin{figure}[h!]
\centering
\includegraphics[width=5in]{race.png}
\caption{Histogram of Racial Diversity}
\label{fig:race}
\end{figure}




\section{Discussion}

Gender and ethnic diversity is important. While not the only factors in a person's outlook, race and gender tend to have a significant impact on a individual perspectives. Through an increase in diversity, a broader sprectrum of viewpoints often becomes available. This can help ensure that technology reflects the needs of society as a whole, rather than a specific, narrow subset.

In my experiences thus far, I have not had any problems with the department at Notre Dame, although I rejected few of the tech clubs I considered joining my Freshman year due to the fact that I was one of only two or three women present at the first meetings. Likewise, I did not have much interest in becoming active in SWE, which is near-exclusively female.

Overall, though it's not perfect, I'm satisfied with Notre Dame's approach to gender and racial imbalances within the engineering school.

\end{document}

